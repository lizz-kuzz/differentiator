\documentclass[12pt, letterpaper, twoside]{article}
\usepackage[T1]{fontenc}
\usepackage[utf8]{inputenc}
\usepackage[russian]{babel}
\usepackage[a1paper]{geometry}
\geometry{papersize={29.7 cm, 25.0 cm}}
\title{ВЗЯТИЕ ПРОИЗВОДНОЙ КРОКОДИЛА}
\author{Очевидно, что Кузнецова Елизавета Юрьевна}
\date{22.11.2022}
\begin{document}
\maketitle
\newpage
Подробное описание взятия производной заданной функции:
\[f(x) ={{({({1} - {x})}^{3})} \cdot {({({x} - {2})}^{3})}}\]\newline\newlineне составит труда, осознать, что РТ лучший факультет, а также, что:\newline\newline\[{({{({({1} - {x})}^{3})} \cdot {({({x} - {2})}^{3})}})}^{'} = {({{({1} - {x})}^{3}})}^{'}\cdot {{({x} - {2})}^{3}} + {{({1} - {x})}^{3}}\cdot {({{({x} - {2})}^{3}})}^{'}\]
\newlineзарубите себе на носу данный факт:\newline\newline\[{({{({x} - {2})}^{3}})}^{'} = ({3})\cdot {{{x} - {2}}}^{{3} - 1} \cdot{({{x} - {2}})}^{'}\]
\newlineданное выражение упрощается очевидным способом(если для вас это не очевидно, это ваши проблемы :) ):\newline\newline\[{({{x} - {2}})}^{'} = {({x})}^{'} - {({2})}^{'}\]
\newlineне требует дальнейших комментариев:\newline
\[{(2)}^{'} = 0 \]

легко зметить, что фопф - х***я, ну а так же, что:\newline
\[{(x)}^{'} = 1 \]

как всем известно, фивты по ночам с фопфами...дифференцируют имеено так:\newline\newline\[{({{({1} - {x})}^{3}})}^{'} = ({3})\cdot {{{1} - {x}}}^{{3} - 1} \cdot{({{1} - {x}})}^{'}\]
\newlineочевидно, что фупм гавно, а также, что:\newline\newline\[{({{1} - {x}})}^{'} = {({1})}^{'} - {({x})}^{'}\]
\newlineне составит труда, осознать, что РТ лучший факультет, а также, что:\newline
\[{(x)}^{'} = 1 \]

зарубите себе на носу данный факт:\newline
\[{(1)}^{'} = 0 \]

Производная заданной функции:
\[f'(x) ={{{{{3} \cdot {({({1} - {x})}^{({3} - {1})})}} \cdot {({0} - {1})}} \cdot {({({x} - {2})}^{3})}} + {{({({1} - {x})}^{3})} \cdot {{{3} \cdot {({({x} - {2})}^{({3} - {1})})}} \cdot {({1} - {0})}}}}\]
\newline\newlineУпростим данного крокодила(что является достаточно очевидной задачей):
\[f'(x) ={{{{3} \cdot {({({1} - {x})}^{2})}} \cdot {({({x} - {2})}^{3})}} + {{({({1} - {x})}^{3})} \cdot {{3} \cdot {({({x} - {2})}^{2})}}}}\]
\newline\newline\end{document}
